\documentclass[12pt]{article}
\usepackage[utf8]{inputenc}
\usepackage[russian]{babel}
\usepackage{multicol}
\begin{document}
\begin{multicols}{2}
из них: \(m=0, 1, 3, 5; -2, -4\). Вычисляя при этих значенях лувую и правую части уранения $(2)$, получаем таблицу 
\begin{tabular}{|c|c|c|c|c|c|c|} %l - по лувому краю с - по правому к - по левому
\hline
$m$ & $0$ & $1$ & $3$ & $5$ & $-2$ & $-4$ \\
$sin$$\frac{f}{r}$
&1 &1 &1 & $\frac{\sqrt{3}}{2}$ &1 &1  \\
\(sin\frac{f}{r}\) & 1& 1& 1& 1& 1& 1 \\
\hline
\end{tabular}
\end{multicols}
\end{document}